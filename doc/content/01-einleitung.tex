\section{Einleitung}
\label{sec:einleitung}

Diese Arbeit untersucht ob tcpdump\footnote{\url{http://www.tcpdump.org/}} zum
Monitoren von HTTP Traffic in einem Hochlastszenario geeignet ist. Der Use
Case ist, dass tcpdump dazu eingesetzt werden soll HTTP Traffic zu überwachen
und zu analysieren. Anhand der gewonnen Daten sollen Entscheidungen getroffen
werden um eine Lastverteilungsinfrastruktur zu steuern. Um dies sinnvoll
einzusetzen muss sichergestellt werden, dass das Monitoring alle oder
mindestens den größten Teil der HTTP Requests aufzeichnen kann.

Um in diesem Kontext tcpdump zu evaluieren wird diese Arbeit zuerst die
Grundlagen von tcpdump im Kapitel \ref{sec:grundlagen} erklärt. Dabei wird
gezeigt wie tcpdump Pakete filtert und wo es dort zu Paketverlusten kommen
kann. Anschließend werden in Kapitel \ref{sec:verwandte-arbeiten} Arbeiten
aufgeführt die sich bereits mit der Performance von tcpdump beschäftigt haben.
Das Kapitel \ref{sec:konzept} beschreibt dann den Versuchsaufbau für diese Arbeit
und was dabei die erwarteten Beobachtungen sind. Die Messergebnisse der
Experimente werden in Kapitel \ref{sec:messungen} vorgestellt und ausgewertet.
Das letzte Kapitel \ref{sec:zusammenfassung} gibt eine Zusammenfassung der
Ergebnisse dieser Arbeit und eine Einschätzung ob tcpdump in dem beschrieben
Use Case sinnvoll eingesetzt werden kann.

Die Beschreibungen und Messungen in dieser Arbeiten basieren alle auf der tcpdump
Version 4.3.0 und der dazu gehörigen libpcap Version 1.3.0. Das Betriebssystem
auf dem tcpdump lief war Centos 5 mit einem Kernel der Version 2.6.18.

Diese Arbeit befasst sich nur mit der Paketaufzeichnung und nicht mit der
Weiterverarbeitung und Auswertung der aufgezeichneten Daten.
