\section{Konzept}
\label{sec:konzept}

Um die Leistung von tcpdump zur Aufzeichnung von HTTP Traffic zu zeigen wurden
im Rahmen dieser Arbeit mehrere Messungen vorgenommen. Dieses Kapitel beschreibt
das Konzept und den Aufbau dieser Messungen. Und begründet die Entscheidungen
bezüglich der einzelne Komponenten.

Für die Messung standen zwei identische Systeme \texttt{ib5} und \texttt{ib6}
zur Verfügung. Dabei wurde \texttt{ib5} als Benchmark Client und \texttt{ib6}
als System unter Last (SUT) genutzt.

\begin{description}
\item [ib5/ib6] \hfill
\begin{description}
\item [CPU] Quad Core Intel Xeon 2.27 GHz
\item [RAM] 6 GB
\item [OS] CentOS 5.8
\item [Kernel] 2.6.18-308.el5
\item [NIC Treiber] bnx2 2.1.11
\end{description}
\end{description}

Als HTTP Benchmark wurde \texttt{wrk}\footnote{\url{https://github.com/wg/wrk}}
Version 4.0.2 auf \texttt{ib5} genutzt. Dieser Benchmark nutzt Multithreading
und Event Loops um eine möglichst hohe Anzahl an HTTP Requests an das SUT zu
schicken. Auf \texttt{ib6} wurde
\texttt{nginx}\footnote{\url{http://nginx.org/}} Version 1.9.15 als HTTP Server
eingesetzt. Zur Aufzeichnung der Netzwerkpakete wurde
\texttt{tcpdump}\footnote{\url{http://www.tcpdump.org/}} Version 4.3.0 und
\texttt{libpcap} Version 1.3.0 verwendet.

Ziel des Messaufbau war es eine möglichst hohe Anzahl an HTTP Requests zu
erzeugen um eine hohe Last auf tcpdump zu generieren. Um dies zu erreichen
wurden minimal HTTP Requests und Responses genutzt. Dazu wurde nginx so
konfiguriert das es auf einen HTTP GET Requests direkt mit einem 204 No Content
Response antwortet (siehe Listing~\ref{lst:nginx} in
Anhang~\ref{sec:konfigurationen}).  Dies entspricht zwar keinem realen
Szenario, allerdings kann eine reale Überlastsituation nicht mit der vorhanden
Hardware simuliert werden. Es ist jedoch vorstellbar das die Anzahl der Pakete
die durch diese Vereinfachung versendet wurden in einem realen Lastszenario
erreicht werden können. Und eine geringere Last keine Probleme für tcpdump
dargestellt hätten.

\begin{lstlisting}[caption={Aufruf von wrk},label=lst:wrk]
wrk -t 4 -c 1024 -d 5m http://ib6
\end{lstlisting}

\begin{table}
  \centering
  \bgroup
  \def\arraystretch{1.3}
  \begin{tabular}{ccccp{7.6cm}}
      & \textbf{Name} & \textbf{snaplen} & \textbf{buffer} & \textbf{filter} \\\hline\hline
      1 & no & \multicolumn{3}{c}{---} \\\hline
      2 & default & 65535 & 2048 & ip dst host 172.16.0.26 and tcp dst port 80 \\\hline
      3 & snaplen & 142 & 2048 & ip dst host 172.16.0.26 and tcp dst port 80 \\\hline
      4 & buffer  & 65535 & 4096 & ip dst host 172.16.0.26 and tcp dst port 80 \\\hline
      5 & snaplen+buffer & 142 & 4096 & ip dst host 172.16.0.26 and tcp dst port 80 \\\hline
      6 & filter  & 142 & 4096 & ip dst host 172.16.0.26 and tcp dst port 80 and \\
        &      & & & 'tcp[((tcp[12:1] \& 0xf0) >> 2):4] = 0x47455420' \\
  \end{tabular}
  \egroup
  \caption{Konfiguration von tcpdump für Messszenarien}\label{tab:szenarien}
\end{table}

Um die in Kapitel~\ref{sec:grundlagen} aufgeführten Einstellungen von tcpdump
zu untersuchen wurde insgesamt 6 Messreihen durchgeführt. Es wurde Untersucht
wie sich tcpdump allgemein auf die Performance des Webservers auswirkt. Und wie
sich die Parameter Größe der aufgezeichneten Pakete (snaplen), Größte des
Paketbuffers (buffer) und der verwendete Paketfilter (filter) auf tcpdump
auswirken. Bei allen Messungen war die nginx Konfiguration gleich (siehe
Listing~\ref{lst:nginx} in Anhang~\ref{sec:konfigurationen}).  Ebenfalls
wurde wrk immer wie in Listing~\ref{lst:wrk} gezeigt aufgerufen. Jede Messung
war 5 Minuten lang und benutzt 4 Threads und 1024 Verbindungen. Die
Konfiguration von tcpdump wurde für jede Messung angepasst (siehe
Tabelle~\ref{tab:szenarien}). Im Folgenden sollen die einzelnen Szenarien
beschrieben werden.

\paragraph{no} Die erste Messreihe ist eine Kontrollmessung ohne tcpdump. Diese
Messung soll die maximale Anzahl der Requests in der Messumgebung ermitteln.
Anhand dieser Messung können dann Aussagen über den Einfluss von tcpdump auf
die Leistung des HTTP Servers getroffen werden.

\paragraph{default} Bei dieser Messungen wurde die Standardparameter von tcpdump
verwendet. Dabei ist die snaplen 65535 Bytes und der buffer 2048 KBytes. Der Filter
beschreibt alle TCP Pakete für die Ziel-Adresse \texttt{172.16.0.26} und dem Ziel-Port
\texttt{80}. Dies umfasst alle eingehenden TCP Pakete für den Webserver.

\paragraph{snaplen} In diesem Szenario wurden die snaplen auf 142 Bytes
reduziert. Die ersten 142 Bytes eines TCP Paketes reichen aus um zu analysieren
ob es sich um ein HTTP Paket handelt.  Dabei ergibt sich 142 Bytes aus der Mac
Header Größe (14 Bytes), der maximalen IP Header Größe (60 Bytes), der
maximalen TCP Header Größe (60 Bytes) und den ersten 8 Bytes der TCP Payload
(HTTP Methode). Diese Aufzeichnungsgröße ist ausreichend um zu analysieren
wieviel HTTP Pakete den Webserver erreichen, was dem untersuchten Use Case
entspricht. Da für den IP und TCP Header die maximalen Größen angenommen
wurden, werden vermutlich die meisten aufgezeichneten Pakete noch weitere
Informationen wie den Request Pfad enthalten.

\paragraph{buffer} In diesem Szenario wurde die Buffergröße auf 4096 Bytes
verdoppelt, dies sollte es tcpdump ermöglichen mehr Pakete für die
Weiterverarbeitung vorzuhalten.

\paragraph{snaplen+buffer} Diese Szenario kombiniert die Szenarien
\texttt{snaplen} und \texttt{buffer} um die Anzahl der Pakete welche in den
tpcdump Buffer passen zu maximieren.

\paragraph{filter} Das letzte Szenario soll die Auswirkung des gewählten
Filters untersuchen. Dazu wurde der Filter noch einmal verfeinert. Der Filter
für Ziel-IP und Ziel-Port akzeptiert sowohl HTTP Requests als auch TCP Pakete
welche zum Verbingungsaufbau und -abbau gesendet werden. Um diese bereits im
Kernel herauszufiltern wurde der Filter um den Ausdruck \texttt{tcp[((tcp[12:1]
\& 0xf0) >> 2):4] = 0x47455420} erweitert. Dieser Filter testet ob die ersten 4
Bytes der TCP Payload \glqq{}\texttt{GET }\grqq{} entsprechen.  Der
Filter akzeptiert also nur Pakete die GET HTTP Requests sind. Außerdem wurden
die snaplen und buffer Größe vom \texttt{snaplen+buffer} Szenario übernommen.

Jede Messung wurde 7 mal wiederholt. Dabei wurden die Ausgaben von
wrk und tcpdump aufgezeichnet und ausgewertet. Die folgenden Metriken
wurden dann untersucht:

\begin{enumerate}
\item Anzahl der gesendeten HTTP Requests (wrk)
\item Anzahl der gefilterten Netzwerkpakete (tcpdump)
\item Anzahl der verlorenen Netzwerkpakete (tcpdump)
\end{enumerate}

Es wurde erwartet, dass anhand der Ergebnisse gezeigt werden kann, dass tcpdump
eine Auswirkung auf die Anzahl der gesendeten HTTP Requests hat. Außerdem
das es bereits in diesem einfach Messaufbau mit nur einem Client zu Paketverlusten
kommt. Diese jedoch durch eine Anpassung der tcpdump Parameter snaplen und buffer
reduziert werden können. Weiterhin ist die Vermutung, dass ein sehr spezifischer
Filter eine weitere Verbesserung ermöglicht. Das folgende Kapitel~\ref{sec:messungen}
beschreibt die Ergebnisse der einzelnen Messungen und wertet diese aus.
