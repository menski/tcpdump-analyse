\section{Verwandte Arbeiten}
\label{sec:verwandte-arbeiten}

In \cite{schneider05da,DBLP:conf/conext/SchneiderW05} wurde untersucht ob sich
nicht spezialisierte Hardware zum Aufzeichnen von Paketen in Gigabit Netzwerken
eignet. Es wurden 4 verschieden Systeme verglichen, welche sich nur in der
verwendeten CPU und dem Betriebssystem unterschieden. Als CPUs wurden AMD
Opteron und Intel Xeon verwendet. Die Betriebssysteme waren Linux 2.6.11 und
FreeBSD 5.4. In den Messungen wurde der Linux Kernel Package Generator
\cite{DBLP:journals/comcom/TurullSO16} genutzt um UDP Pakete zu generieren. Die
Paketgrößen wurde anhand einer Verteilung generiert, welche aus einer 24h
Aufzeichnung abgeleitet wurde. Die Ergebnisse der Experimente zeigten, dass mit
steigender Paketrate die CPU Auslastung und der Paketverlust bei der
Aufzeichnung steigt. Außerdem zeigte sich das die Kombination aus FreeBSD und
AMD Opteron den geringsten Paketverlust aufzeigte. In einer anknüpfenden Arbeit
\cite{DBLP:conf/pam/SchneiderWF07} wurden Ansätze beschrieben wie man höhere
Netzwerkarten wie 10-Gigabit Ethernet auf mehrere 1-Gigabit Interfaces
verteilen kann. Dadurch kann der erhöhte Netzwerkverkehr immer noch effizient
aufgezeichnet werden.

Den Einfluss von tcpdump auf die Leistung eines parallel laufenden Dienstes,
wie zum Beispiel einem Webserver, wurde in \cite{DBLP:conf/globecom/ChenCCM11}
untersucht. Es zeigte sich eine starke Beeinträchtigung, wenn sich
tcpdump und der Webserver eine CPU geteilt haben. Allerdings war auch bei
getrennten CPUs die Leistung des Webservers beeinträchtigt.

Technologien wie Intel Data Plane Development
Kit\footnote{\url{http://dpdk.org/}} (DPDK) and
WireCAP\footnote{\url{http://wirecap.fnal.gov/}} \cite{DBLP:conf/imc/WuD14}
untersuchen neue Ansätze um die Paketverarbeitung und -aufzeichnung zu
optimieren.  Dabei stehen Multi-Core Unterstützung und optimierte
Speicherverwaltung im Vordergrund.  Dabei können in Hochlastszenarien
Verbesserungen zu normalen libpcap Anwendungen erzielt werden.

Die existierende Literatur zeigt, dass das Problem des Paketverlustes bei
der Aufzeichnung von Netwerkverkehr unter hoher Last ein bekanntes Problem
ist. Dabei haben viele Faktoren einen Einfluss auf die Paketaufzeichnung. Dies
umfasst die verwendete Hardware, wie Netzwerkkarten, CPU und Festplatten, aber
auch das verwendete Betriebssystem und Software. Da diese Arbeit sich explizit
mit tcpdump/libpcap auf einem gegeben System befasst, wird im folgenden Kapitel
ein Konzept vorgestellt um die Auswirkungen der in Kapitel \ref{sec:grundlagen}
vorgestellten Aspekte von tcpdump/libpcap zu untersuchen.

