\section{Zusammenfassung}
\label{sec:zusammenfassung}

In dieser Arbeit wurde untersucht, wie tcpdump/libpcap Netzwerkpakete filtert
und aufzeichnet. Dazu wurde zu erst gezeigt, wie ein TCP Paket durch den Linux
Netzwerkstack verarbeitet wird. Dabei wurden die möglichen Ursachen für den
Paketverlust von tcpdump erklärt. Anschließend wurde ein Messkonzept erstellt,
um die Auswirkung dieser Punkte zu untersuchen. Die daraus resultierenden
Messungen bestätigten die Erwartung, dass diese Parameter entscheidend für die
Anzahl der verlorenen Pakete sind.

Es zeigte sich das es sehr wahrscheinlich zu Paketverlusten kommen kann, wenn
tcpdump zum Aufzeichnen von sehr hohem Netzwerkverkehr genutzt werden soll.
Allerdings kann diese Problem durch gezielte Auswahl der tcpdump Einstellungen
reduziert werden. Wie spezifisch die Einstellungen gewählt werden können hängt
vom jeweiligen Use Case ab. So konnte für den hier diskutierten Einsatzzweck,
der HTTP Lastverteilung, sehr spezifische Einstellungen vorgenommen werden.
Soll jedoch zum Beispiel der komplette TCP Verkehr von mehreren Anwendungen
aufgezeichnet werden, zum Beispiel für Intrusion Detection, dann muss ein recht
grober Filter gewählt werden und ein großer Teil des Paketes aufgezeichnet
werden.

Zusammenfassend konnte gezeigt werden, dass tcpdump zum Aufzeichnen von HTTP
Requests genutzt werden kann, sofern die Einstellungen auf diesen speziellen
Einsatzzweck optimiert werden. Außerdem sollte tcpdump nicht parallel auf dem
System betrieben wird, welches unter Last steht, da es einen Einfluss auf die
Leistung des Systems hat. Daher wäre es zu empfehlen das tcpdump Monitoring auf
einem dedizierten System durchzuführen. Dies ist auch deshalb zu empfehlen,
weil eine Weiterverarbeitung der tcpdump Aufzeichnung nötig ist, welche in
dieser Arbeit nicht betrachtet wurde.

